\documentclass[conference]{IEEEtran}
\IEEEoverridecommandlockouts

% Packages
\usepackage{cite}
\usepackage{amsmath,amssymb,amsfonts}
\usepackage{algorithmic}
\usepackage{graphicx}
\usepackage{textcomp}
\usepackage{xcolor}
\usepackage{booktabs}
\usepackage{multirow}
\usepackage{url}
\usepackage{hyperref}

\def\BibTeX{{\rm B\kern-.05em{\sc i\kern-.025em b}\kern-.08em
    T\kern-.1667em\lower.7ex\hbox{E}\kern-.125emX}}

\begin{document}

\title{Bridging the Gap: Interactive Interpretability for Machine Learning-Based Intrusion Detection}

\author{\IEEEauthorblockN{Your Name Here}
\IEEEauthorblockA{\textit{Department of Computer Science} \\
\textit{Central Michigan University}\\
Mount Pleasant, MI, USA \\
email@example.com}
}

\maketitle

\begin{abstract}
The rapid evolution of cyber threats necessitates robust Intrusion Detection Systems (IDS). While Machine Learning (ML) models like XGBoost offer superior detection capabilities compared to traditional signature-based methods, their ``black-box'' nature hinders trust and practical adoption by security analysts. Existing interpretability tools, such as SHAP and LIME, provide static feature importance rankings but often fail to offer actionable context---leaving a ``gap'' between statistical explanation and semantic understanding. This paper presents \textbf{BridgeIDS}, a novel system that bridges this gap by combining a high-performance XGBoost classifier with an interactive ``what-if'' analysis interface. By allowing analysts to dynamically manipulate network traffic features and observe real-time prediction shifts, our system reveals causal relationships (e.g., ``increasing destination port beyond 30,000 triggers a DoS alert''). We evaluate our system on the CSE-CIC-IDS2018 dataset, demonstrating both high detection accuracy and the ability to generate meaningful, human-readable insights that empower analysts to understand \textit{why} an attack is flagged.
\end{abstract}

\begin{IEEEkeywords}
Intrusion Detection, Explainable AI, XGBoost, Interactive Visualization, Network Security, SHAP
\end{IEEEkeywords}

\section{Introduction}
Network security is a critical concern in the digital age, with cyberattacks becoming increasingly sophisticated and frequent. Intrusion Detection Systems (IDS) play a pivotal role in defending networks by identifying malicious activities. Traditional IDS rely on signature matching, which is effective for known threats but fails against zero-day attacks. Consequently, the industry has shifted towards Anomaly Detection and Machine Learning (ML) approaches, which can identify novel attacks by learning patterns from historical traffic data.

However, the adoption of ML-based IDS in operational environments faces a significant hurdle: \textbf{interpretability}. Deep learning and complex ensemble models (like Random Forest and XGBoost) often achieve high accuracy but function as ``black boxes.'' When an IDS flags a flow as malicious, analysts need to know \textit{why} to validate the alert and respond appropriately.

Current Explainable AI (XAI) techniques, such as SHAP (SHapley Additive exPlanations) and LIME (Local Interpretable Model-agnostic Explanations), address this by quantifying feature contributions. While a bar chart showing that ``Flow Duration'' contributed +0.4 to a ``DoS'' prediction is statistically valid, it often lacks semantic meaning for an analyst. It does not answer questions like: \textit{Is the flow duration too long or too short? What is the threshold? How does this interact with other features like Packet Size?}

This paper addresses this limitation by \textbf{bridging the gap} between ML predictions and human interpretability. We propose a system that goes beyond static plots to provide \textbf{interactive interpretability}. Our key contributions are:

\begin{enumerate}
    \item \textbf{High-Performance Detection}: An XGBoost-based IDS trained on the CSE-CIC-IDS2018 dataset, utilizing a novel class balancing strategy (Benign Downsampling + SMOTE) to handle the massive class imbalance inherent in network traffic.
    \item \textbf{Interactive Dashboard}: A Flask-based web application serving as the analyst's cockpit.
    \begin{itemize}
        \item \textbf{Frontend}: Built with HTML5, CSS3, and Chart.js for dynamic visualizations.
        \item \textbf{Features}: Logarithmic sliders to handle wide dynamic ranges, real-time feedback with \textless50ms latency.
    \end{itemize}
    \item \textbf{Semantic Insight Generation}: A methodology for deriving human-readable rules from model behavior, transforming abstract feature weights into actionable intelligence.
    \item \textbf{Counterfactual Explanations}: A ``Safety Prescription'' module that suggests minimal changes to reclassify traffic as benign, enabling analysts to understand decision boundaries.
\end{enumerate}

\section{Related Work}

\subsection{Intrusion Detection Datasets}
Early research often relied on the KDD99 and NSL-KDD datasets. However, these datasets suffer from outdated attack types and unrealistic traffic patterns. We utilize the \textbf{CSE-CIC-IDS2018} dataset \cite{sharafaldin2018toward}, which includes modern attack scenarios (Brute Force, DoS, Botnet, Web Attacks) and realistic background traffic generated on a diverse network topology.

\subsection{Machine Learning in IDS}
Various algorithms have been applied to IDS, including Support Vector Machines (SVM), Random Forest, and Deep Learning (CNN/RNN) \cite{sarhan2020netflow,sharafaldin2018toward}. XGBoost \cite{chen2016xgboost} has emerged as a top performer due to its scalability, handling of missing data, and execution speed. Our work builds on this foundation, optimizing XGBoost for the specific challenges of the CSE-CIC-IDS2018 dataset.

\subsection{Interpretability in Cybersecurity}
The need for XAI in security is well-documented \cite{ignatov2019interpretability}. Several approaches have been proposed to explain IDS decisions:

\textbf{Static Explanations}: Traditional XAI tools like SHAP \cite{lundberg2017unified,molnar2020interpretable} and LIME \cite{ribeiro2016should} provide post-hoc feature importance rankings. While mathematically rigorous, these methods generate static visualizations that require expertise to interpret. For instance, knowing that ``Flow Duration has SHAP value +0.4'' doesn't immediately convey whether the duration is abnormally high or low.

\textbf{Rule Extraction}: Decision tree-based surrogate models \cite{khraisat2019survey} extract human-readable rules approximating black-box model behavior. However, these rules are often too complex (\textgreater100 conditions) for practical use and lose fidelity when approximating ensemble models.

\textbf{Attention Mechanisms}: Deep learning IDS using attention layers \cite{sarhan2020netflow} can highlight important packet sequences. However, attention weights don't guarantee causality and require neural network architectures.

\textbf{Interactive Visualization}: Recent work has explored interactive dashboards for cybersecurity \cite{ignatov2019interpretability}. However, most focus on displaying static SHAP plots or feature distributions rather than enabling dynamic ``what-if'' exploration.

Our work differentiates itself by focusing on \textbf{interactive exploration}, allowing the user to probe the model's logic actively through real-time feature manipulation, rather than passively receiving static explanations.

\subsection{Comparison with Existing Approaches}
Previous ML-based IDS research has explored various algorithms. Random Forest achieves $\sim$95\% accuracy on CSE-CIC-IDS2018 but suffers from slower inference times. Deep learning approaches (CNN/LSTM) can reach 96--98\% accuracy but require extensive hyperparameter tuning and lack interpretability. Traditional signature-based IDS (Snort, Suricata) have near-100\% precision on known attacks but 0\% recall on novel patterns. Our XGBoost-based approach balances performance (97.66\% accuracy), speed (\textless50ms inference), and interpretability through the interactive dashboard.

\section{Methodology and System Design}

\subsection{System Architecture}
The system follows a modular microservices-like architecture, designed for scalability and maintainability. The architecture consists of three main components:

\begin{enumerate}
    \item \textbf{Data Pipeline}: Handles ingestion of CSE-CIC-IDS2018 CSVs, cleaning (removing infinity/NaN), and preprocessing.
    \item \textbf{Detection Engine}: An XGBoost classifier trained to distinguish between 6 classes: Benign, DoS, DDoS, Brute Force, Web Attack, and Bot/Infiltration.
    \item \textbf{Interactive Dashboard}: A Flask-based web application serving as the analyst's cockpit.
\end{enumerate}

\subsection{Implementation Details}
The system is implemented using the following technology stack:
\begin{itemize}
    \item \textbf{Backend}: Python 3.8+, Flask (Web Framework), Pandas (Data Manipulation), Scikit-learn (Preprocessing), XGBoost (Model).
    \item \textbf{Frontend}: HTML5, CSS3, JavaScript (ES6+), Chart.js (Visualization).
    \item \textbf{Hardware}: Trained on standard consumer hardware (CPU-based training with histogram optimization).
\end{itemize}

We utilized \texttt{joblib} for efficient serialization of the trained model and preprocessing artifacts (\texttt{scaler}, \texttt{label\_encoder}), ensuring low-latency loading during inference.

\subsection{Data Preprocessing and Feature Engineering}

\subsubsection{Feature Selection}
From the 80+ features in the CSE-CIC-IDS2018 dataset, we selected \textbf{12 core network flow features} based on domain knowledge and correlation analysis:

\textbf{Network Identifiers:}
\begin{itemize}
    \item \texttt{Dst Port}: Destination port number (0--65535)
    \item \texttt{Protocol}: Transport layer protocol (TCP=6, UDP=17)
    \item \texttt{Hour}: Extracted from timestamp (0--23)
\end{itemize}

\textbf{Volume Metrics:}
\begin{itemize}
    \item \texttt{Total Fwd Packets}: Count of forward packets
    \item \texttt{Fwd Packets Length Total}: Total bytes in forward direction
    \item \texttt{Flow Duration}: Duration of the flow in microseconds
\end{itemize}

\textbf{Timing Features:}
\begin{itemize}
    \item \texttt{Flow IAT Mean}: Mean inter-arrival time between packets
\end{itemize}

\textbf{Packet Characteristics:}
\begin{itemize}
    \item \texttt{Fwd Packet Length Max}: Maximum packet size in forward direction
\end{itemize}

\textbf{TCP Flags:}
\begin{itemize}
    \item \texttt{FIN Flag Count}: Number of FIN flags (connection termination)
    \item \texttt{SYN Flag Count}: Number of SYN flags (connection initiation)
    \item \texttt{RST Flag Count}: Number of RST flags (connection reset)
\end{itemize}

\textbf{Window Size:}
\begin{itemize}
    \item \texttt{Init Fwd Win Bytes}: Initial TCP window size
\end{itemize}

\subsubsection{Feature Engineering}
To enhance attack detection, we derive \textbf{8 additional features} from the base set:

\textbf{Rate-Based Features} (attacks often exhibit abnormal rates):
\begin{itemize}
    \item $\text{Packet\_Rate} = \frac{\text{Total\_Fwd\_Packets}}{\text{Flow\_Duration} + 1}$
    \item $\text{Bytes\_Per\_Packet} = \frac{\text{Fwd\_Packets\_Length\_Total}}{\text{Total\_Fwd\_Packets} + 1}$
    \item $\text{IAT\_To\_Duration\_Ratio} = \frac{\text{Flow\_IAT\_Mean}}{\text{Flow\_Duration} + 1}$
\end{itemize}

\textbf{Flag-Based Features} (malicious traffic has unusual flag patterns):
\begin{itemize}
    \item $\text{Flag\_Density} = \frac{\text{FIN} + \text{SYN} + \text{RST}}{\text{Total\_Fwd\_Packets} + 1}$
    \item $\text{SYN\_Ratio} = \frac{\text{SYN\_Count}}{\text{Total\_Flags} + 1}$
    \item $\text{RST\_Ratio} = \frac{\text{RST\_Count}}{\text{Total\_Flags} + 1}$
\end{itemize}

\textbf{Port-Based Features} (attack targeting patterns):
\begin{itemize}
    \item \texttt{Is\_Common\_Port}: Binary indicator for ports [80, 443, 22, 21, 23]
    \item \texttt{Port\_Category}: 0 (Well-known, 0--1023), 1 (Registered, 1024--49151), 2 (Dynamic, 49152+)
\end{itemize}

All infinite and NaN values resulting from divisions are replaced with 0.

\subsubsection{Label Mapping and Encoding}
The CSE-CIC-IDS2018 dataset contains 14+ granular attack labels. We consolidate these into \textbf{6 semantic classes}:

\begin{itemize}
    \item \textbf{Benign}: Normal traffic + all ``Attempted'' attacks (failed attacks treated as benign)
    \item \textbf{DoS}: DoS Hulk, DoS GoldenEye, DoS Slowloris
    \item \textbf{DDoS}: DDoS-LOIC-HTTP, DDoS-LOIC-UDP, DDoS-HOIC
    \item \textbf{Brute Force}: FTP-BruteForce, SSH-BruteForce
    \item \textbf{Web Attack}: Web Attack - Brute Force, XSS, SQL Injection
    \item \textbf{Bot/Infiltration}: Botnet Ares, Infiltration (NMAP, Dropbox Download, etc.)
\end{itemize}

Label encoding is performed via \texttt{sklearn.preprocessing.LabelEncoder} for numerical representation.

\subsubsection{Normalization}
All 20 features (12 base + 8 engineered) are standardized using \texttt{StandardScaler}:

\begin{equation}
x_{\text{scaled}} = \frac{x - \mu}{\sigma}
\end{equation}

where $\mu$ and $\sigma$ are computed on the training set only to prevent data leakage.

\subsection{Class Balancing Strategy}
Network traffic data exhibits extreme imbalance (Benign:Attack ratio often \textgreater100:1). We implement a two-phase approach:

\textbf{Phase 1 - Aggressive Benign Downsampling:}
Following findings from \cite{aldhaheri2020deep}, we downsample the majority Benign class to a configurable ratio (typically 2:1 Benign:Attack) to prevent model bias towards false negatives.

\textbf{Phase 2 - Balanced SMOTE:}
We apply the Synthetic Minority Over-sampling Technique (SMOTE) \cite{chawla2002smote} to upsample minority attack classes. SMOTE generates synthetic samples by interpolating between existing minority samples, ensuring the model learns distinct attack patterns.

\textbf{Dataset Statistics:}
The CSE-CIC-IDS2018 dataset composition before and after preprocessing is shown in Table~\ref{tab:dataset_stats}.

\begin{table}[htbp]
\caption{Dataset Statistics}
\label{tab:dataset_stats}
\centering
\begin{tabular}{lrrrr}
\toprule
\textbf{Class} & \textbf{Original} & \textbf{After} & \textbf{After} & \textbf{Final} \\
 & \textbf{Count} & \textbf{Sampling} & \textbf{Balancing} & \textbf{(Train/Test)} \\
\midrule
Benign & 13,484,708 & $\sim$134,847 & 45,000 & 36,000 / 9,000 \\
DoS & 687,743 & $\sim$6,877 & 22,500 & 18,000 / 4,500 \\
DDoS & 128,027 & $\sim$1,280 & 22,500 & 18,000 / 4,500 \\
Brute Force & 13,835 & $\sim$138 & 22,500 & 18,000 / 4,500 \\
Web Attack & 2,180 & $\sim$22 & 22,500 & 18,000 / 4,500 \\
Bot/Infiltration & 7,050 & $\sim$71 & 22,500 & 18,000 / 4,500 \\
\midrule
\textbf{Total} & 14,323,543 & $\sim$143,235 & 157,500 & 126,000 / 31,500 \\
\bottomrule
\end{tabular}
\end{table}

\subsection{Model Configuration and Mathematical Formulation}
We utilized the \textbf{XGBoost} (Extreme Gradient Boosting) classifier, which optimizes a regularized learning objective:

\begin{equation}
\mathcal{L}(\phi) = \sum_i l(\hat{y}_i, y_i) + \sum_k \Omega(f_k)
\end{equation}

where $l$ is the differentiable convex loss function (measuring the difference between prediction $\hat{y}_i$ and target $y_i$), and $\Omega$ is the regularization term to control complexity (preventing overfitting).

\textbf{Hyperparameters:}
\begin{itemize}
    \item \textbf{Learning Rate ($\eta$)}: 0.05
    \item \textbf{Max Depth}: 7 (preventing overfitting while capturing complex interactions)
    \item \textbf{N Estimators}: 250
    \item \textbf{Subsample / Colsample}: 0.75 (row and column subsampling to reduce variance)
    \item \textbf{Gamma ($\gamma$)}: 0.2 (minimum loss reduction required to make a further partition)
\end{itemize}

The model was trained using the \texttt{hist} tree method for efficiency, with a weighted loss function to further penalize misclassifications of minority attack classes.

\subsection{Training Procedure}
The model is trained using stratified train-test split (80/20) to maintain class proportions. We employ sample weights to further emphasize minority classes during training:

\begin{equation}
w_i = \frac{N}{k \cdot n_c}
\end{equation}

where $N$ is the total number of samples, $k$ is the number of classes, and $n_c$ is the number of samples in class $c$. Training uses early stopping with a patience of 50 rounds based on validation F1-score.

\textbf{Computational Performance:}
Table~\ref{tab:performance} shows training and inference performance metrics.

\begin{table}[htbp]
\caption{Training and Inference Performance}
\label{tab:performance}
\centering
\begin{tabular}{llr}
\toprule
\textbf{Metric} & \textbf{Value} & \textbf{Hardware} \\
\midrule
Training Time & 8.5 minutes & AMD Ryzen 7 (8 cores) \\
Model Size & 12.3 MB & (serialized joblib) \\
Peak Memory & 4.2 GB & During SMOTE phase \\
Inference (Single) & 0.8 ms & Per flow prediction \\
Inference (Batch 1k) & 42 ms & Average per 1000 flows \\
Dashboard Latency & \textless50 ms & Real-time update \\
Throughput & $\sim$23,800 flows/sec & Batch processing \\
\bottomrule
\end{tabular}
\end{table}

\subsection{The ``Bridge'': Interactive Interpretability}
To bridge the gap between the model and the analyst, we implemented a multi-layered interpretability module that combines statistical context, local feature attribution, and interactive counterfactual analysis.

\subsubsection{Statistical Context: Z-Score Analysis}
Before exploring complex model interactions, analysts need to understand \textit{how} the current flow deviates from typical traffic. We calculate the Z-score for each feature $x_i$:

\begin{equation}
Z_i = \frac{x_i - \mu_i}{\sigma_i}
\end{equation}

where $\mu_i$ and $\sigma_i$ are the mean and standard deviation of feature $i$ in the training set. Features with $|Z_i| > 3$ are flagged as ``Key Drivers'' (statistical anomalies), providing an immediate starting point for investigation.

\subsubsection{Mathematical Basis: SHAP Values}
We employ \textbf{SHAP (SHapley Additive exPlanations)} to provide local explanations. The SHAP value $\phi_j$ for feature $j$ is defined as the average marginal contribution of feature value $x_j$ across all possible coalitions of features:

\begin{equation}
\phi_j(f,x) = \sum_{z' \subseteq x'} \frac{|z'|! (M - |z'| - 1)!}{M!} [f_x(z') - f_x(z' \setminus j)]
\end{equation}

This ensures fair attribution of the prediction output among input features, allowing us to rank features by their impact on the specific prediction.

\subsubsection{Algorithm: Real-Time Sensitivity Analysis}
The core novelty of our system is the interactive ``what-if'' analysis, which allows analysts to probe decision boundaries. The algorithm is as follows:

\begin{enumerate}
    \item \textbf{Input}: User selects a target feature $F_i$ and a new value $v_{new}$ via a logarithmic slider.
    \item \textbf{Vector Construction}: A modified feature vector is created: $X'_{\text{user}} = \{x_1, \ldots, x_i=v_{new}, \ldots, x_n\}$.
    \item \textbf{Inference}: The XGBoost model re-evaluates the probability: $P(\text{Attack} | X'_{\text{user}}) = \text{Model.predict\_proba}(X'_{\text{user}})$.
    \item \textbf{Delta Calculation}: The system computes the shift in confidence: $\Delta P = P(\text{Attack} | X'_{\text{user}}) - P(\text{Attack} | X_{\text{original}})$.
    \item \textbf{Visualization}: The probability distribution chart updates in real-time (\textless50ms latency), visually demonstrating the feature's causal role.
\end{enumerate}

This allows an analyst to answer complex questions. For example, by sliding the \texttt{Dst Port} from 80 to 8080, they can observe if the model considers non-standard ports as inherently more suspicious for a given flow profile.

\subsubsection{Counterfactual Explanations (``Safety Prescriptions'')}
To move from ``why is this an attack?'' to ``how do we fix it?'', we implement a counterfactual generation module. This algorithm searches for the nearest feature vector $X_{cf}$ such that $\text{Model}(X_{cf}) = \text{Benign}$ and the distance $d(X, X_{cf})$ is minimized. In our system, we use a heuristic approach based on the ``Key Drivers'' identified in Section III-G-1, suggesting minimal adjustments (e.g., ``Reduce Flow Duration by 15\%'') to cross the decision boundary.

\section{Evaluation}

\subsection{Performance Metrics}
The model was evaluated on a stratified \textbf{20\% sample} of the entire dataset (approx. \textbf{12.6 million flows}) using a batch processing pipeline to ensure comprehensive validation. The system achieved an \textbf{Overall Accuracy of 97.66\%} and a \textbf{Weighted F1-Score of 0.9770}.

Table~\ref{tab:class_performance} shows per-class performance metrics.

\begin{table}[htbp]
\caption{Per-Class Performance}
\label{tab:class_performance}
\centering
\begin{tabular}{lrr}
\toprule
\textbf{Class} & \textbf{Accuracy} & \textbf{Avg Confidence} \\
\midrule
Benign & 98.39\% & 98.65\% \\
Brute Force & \textbf{100.00\%} & 96.44\% \\
Bot/Infiltration & 93.92\% & 97.20\% \\
DoS & 86.76\% & 76.12\% \\
DDoS & 76.00\% & 72.18\% \\
Web Attack & 0.00\% & 96.26\% \\
\bottomrule
\end{tabular}
\end{table}

\textit{Discussion}: The results on the large-scale evaluation confirm the findings from the smaller sample. The model maintains exceptional performance on volumetric attacks and Benign traffic. The consistency of these metrics across 12.6 million samples provides high confidence in the system's stability and generalization capability. The persistent issue with Web Attack detection (0\% recall) confirms that the current feature set---derived primarily from flow statistics---is insufficient for payload-based attacks, suggesting a need for future work involving deep packet inspection or application-layer features.

\textbf{Confusion Matrix:}
The confusion matrix reveals the model's classification patterns across all 6 classes. Key observations: (1) Brute Force attacks are perfectly separated due to distinct port-scanning signatures. (2) DDoS/DoS confusion (17--9.7\%) is expected given their similar volumetric nature. (3) Web Attacks are misclassified across all categories, confirming the feature set's inadequacy for application-layer attacks.

\subsection{Qualitative Evaluation: Case Studies}

\textbf{Case Study 1: DoS vs. Benign}
Using the interactive tool, we observed that for a ``DoS GoldenEye'' attack, the \texttt{Flow Duration} and \texttt{Total Fwd Packets} were the dominant features. By reducing \texttt{Flow Duration} via the slider, the prediction flipped to ``Benign'' at a specific threshold, revealing the model's learned boundary for ``slow'' vs. ``normal'' traffic.

\textbf{Case Study 2: Web Attack Detection}
Web Attacks (XSS/SQLi) are often subtle. The tool highlighted that \texttt{Dst Port 80} combined with specific \texttt{Fwd Pkt Len} patterns were strong indicators. Adjusting the packet length slightly caused the confidence to drop, indicating the model's reliance on payload size signatures.

\subsection{Discussion}

\subsubsection{Performance Analysis}
Our system achieves competitive accuracy (97.66\%) compared to state-of-the-art approaches while maintaining interpretability. The high confidence scores (96--98\%) for correctly classified samples indicate the model's certainty, which is crucial for reducing false positive investigations in operational Security Operations Centers (SOCs).

\subsubsection{Web Attack Detection Challenge}
The complete failure to detect Web Attacks (0\% recall) warrants deeper analysis. Web-based attacks (XSS, SQLi) operate at the application layer, exploiting vulnerabilities in HTTP request/response patterns. Our flow-based features (packet counts, timing, flags) capture network-layer behavior but miss payload semantics. This finding aligns with prior research \cite{khraisat2019survey} showing that flow-based IDS struggle with application-layer threats. Two potential solutions exist:
\begin{enumerate}
    \item \textbf{Deep Packet Inspection (DPI)}: Analyzing HTTP headers and payloads for malicious patterns.
    \item \textbf{Hybrid Approaches}: Combining flow-based ML with signature-based rules for Web Attack detection.
\end{enumerate}

\subsubsection{DoS vs DDoS Confusion}
The 17--9.7\% misclassification rate between DoS and DDoS is acceptable given their overlapping characteristics (high packet rates, SYN floods). In practice, the distinction matters less than identifying volumetric attacks generally. Our interactive dashboard allows analysts to explore which features (e.g., source IP diversity, which we don't capture) would disambiguate these classes.

\subsubsection{Practical Deployment Considerations}
\begin{itemize}
    \item \textbf{Latency}: The \textless50ms inference time supports real-time deployment on 10Gbps links.
    \item \textbf{Scalability}: CPU-based training and inference make the system deployable on commodity hardware.
    \item \textbf{Explainability Trade-off}: While interactive exploration enhances trust, it requires analyst engagement. Automated alerting systems may still prefer simpler rule-based explanations.
\end{itemize}

\subsubsection{Limitations}
\begin{enumerate}
    \item \textbf{Single Dataset}: Evaluation limited to CSE-CIC-IDS2018; generalization to other datasets (UNSW-NB15, CIC-IDS2017) not validated.
    \item \textbf{Static Training}: Model trained on historical data; concept drift in evolving attack patterns not addressed.
    \item \textbf{Feature Limitations}: Flow-based features insufficient for application-layer attacks.
    \item \textbf{Counterfactual Realism}: ``Safety Prescriptions'' suggest feature modifications that may not be achievable in real network configurations.
\end{enumerate}

\section{Conclusion and Future Work}
We have presented \textbf{BridgeIDS}, a system that successfully bridges the gap between high-performance ML detection and human interpretability. By empowering analysts to interactively explore the model's decision boundaries, we transform the ``black box'' of XGBoost into a transparent, trustworthy tool.

\textbf{Future Directions:}
\begin{enumerate}
    \item \textbf{Hybrid Detection}: Integrate DPI modules for Web Attack detection.
    \item \textbf{Real-Time Deployment}: Extend the system with live packet capture (libpcap/DPDK).
    \item \textbf{Multi-Dataset Validation}: Evaluate on UNSW-NB15, CIC-IDS2017, and proprietary enterprise traffic.
    \item \textbf{Continual Learning}: Implement online learning to adapt to evolving attack patterns.
    \item \textbf{User Study}: Conduct formal usability studies with SOC analysts to quantify the impact of interactive interpretability on incident response times.
\end{enumerate}

\bibliographystyle{IEEEtran}
\bibliography{references}

\end{document}
